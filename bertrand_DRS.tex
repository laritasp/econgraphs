\documentclass[a4paper, 11pt]{article}
\usepackage[utf8]{inputenc}
\usepackage[spanish]{babel}
\usepackage[dvipsnames]{xcolor}
 \usepackage[makeroom]{cancel}
\usepackage{framed, hyperref}
\usepackage{enumitem}
\usepackage{amsmath, amsfonts, amsthm, amssymb}
\usepackage[margin=1.5cm]{geometry}  % margins
\renewcommand{\baselinestretch}{1.15}  % interlineado

\def\Frac#1#2{\color{black} #1 \above 0.4pt \color{red} #2}

% Header and footer
\usepackage{fancyhdr}
\pagestyle{fancy}
\lhead{\textsc{Bertrand Oligopoly with DRS}}
\rhead{\textsc{Lara Sánchez Peña}}
% \lfoot{}
% \rfoot{}

% Gráficos
%\usepackage{graphicx}
%\graphicspath{{./tp1_extra_material/}}

% Para poner código de R
\usepackage{listings}
\definecolor{codegreen}{rgb}{0,0.6,0}
\definecolor{codebackcolour}{rgb}{0.95,0.95,0.92}
\lstdefinestyle{codestyle}{
    backgroundcolor=\color{codebackcolour},   
    commentstyle=\color{codegreen},
    basicstyle=\ttfamily\footnotesize,
    breakatwhitespace=false,         
    breaklines=true,                 
    captionpos=b,                    
    keepspaces=true,                 
    numbers=left,                    
    numbersep=5pt,                  
    showspaces=false,                
    showstringspaces=false,
    showtabs=false,                  
    tabsize=2
}
\lstset{style=codestyle}

% 
\theoremstyle{definition}
\newtheorem{exercise}{Exercise}
\newtheorem{solution}{Solution}
%\theoremstyle{}
\newtheorem{suggestion}{Suggestion}

% \usepackage{mdframed}
% \newmdtheoremenv{solution}{Solution}

% Frame solutions
% \newenvironment{solu}
% {%
% \begin{framed}
%   \begin{solution}
%   }%
%     {%     
%   \end{solution}
% \end{framed}
% }

%   Hide solutions
\newif\ifhideproofs
%\hideproofstrue %uncomment to hide proofs

\ifhideproofs
\usepackage{environ}
\NewEnviron{hide}{}
\let\solution\hide
\let\endsolution\endhide
\fi

\fancypagestyle{plain}{}

%Graphs and stuff
\usepackage{amssymb}
\usepackage{tikz}
\usepackage{pgfplots}
\usepackage{mathtools,amssymb}
\usepackage{xcolor}
\pgfplotsset{compat=1.7}
\usepackage{pst-func}
\usepackage{pstricks}
\usepackage{pst-plot}

% Assorted shortcuts
\newcommand{\cmc}{\overset{m.c.}{\rightarrow}}
\newcommand{\p}[1]{\left(#1\right)}
\newcommand{\cor}[1]{\left[#1\right]}
\newcommand{\lla}[1]{\left\{#1\right\}}
\newcommand{\eps}{\varepsilon}
\newcommand{\lol}{\mathcal{L}}
\newcommand{\RR}{\mathbb{R}}
\newcommand{\QQ}{\mathbb{Q}}
\newcommand{\NN}{\mathbb{N}}

\begin{document}
%\title{} 
%\author{}

%\maketitle 




\begin{solution} %solution exercise 6

 $N$ firms compete in prices in a market with demand $D(p)$. Each firm $i$ has a cost function $C(q_i)$, con $C''(q_i) > 0$, \, $i = 1, \cdots , N$.
Is there an equilibrium in which the resulting market price would be as if firms where price takers?   

Let $\Pi_1$ and $\Pi_N$ be the monopoly and oligopoly benefits, respectively. If there exists $\widetilde{p} \in [p_1,p_N]$ there is a continuum of equilibria.


\pgfmathdeclarefunction{pii}{4}{%
  \pgfmathparse{#4*(#1-#2)^2+#3}%
}
\begin{center}
\begin{tikzpicture}
\begin{axis}[
  no markers, 
  domain=0:4, 
  samples=100,
  ymin=0,
  axis lines*=left, 
  xlabel=$p$,
  ylabel=$\Pi$,
  every axis y label/.style={at=(current axis.above origin),anchor=south},
  every axis x label/.style={at=(current axis.right of origin),anchor=west},
  height=5cm, 
  width=12cm,
  xtick=\empty, 
  ytick=\empty,
  enlargelimits=false, 
  clip=false, 
  axis on top,
  grid = major,
  %hide y axis
  ]
%\addplot [fill=cyan!20, draw=none, domain=4.5:6] {gauss(x,3,1)} \closedcycle;
\draw [yshift=-1cm, latex-latex](axis cs:0.17157, 0) -- node [fill=white, fill opacity=0.4] {$(1)$} (axis cs:1.05051, 0);
\draw [yshift=-1cm, latex-latex](axis cs:1.05051, 0) -- node [fill=white, fill opacity=0.4] {$(2)$} (axis cs:1.87868, 0);
\draw [yshift=-1cm, latex-latex](axis cs:1.87868, 0) -- node [fill=white, fill opacity=0.4] {$(3)$} (axis cs:3.5, 0);
%pi_n
 \addplot [very thick,cyan!50!black] {pii(x, 3, 2,-0.25)} node[above,pos=1]{$\Pi_N$};
 %pi_1
 \addplot [domain=0.5:4, very thick,red!50!black] {pii(x, 3.5, 3,-0.5)}node[above,pos=1]{$\Pi_1$};

\pgfmathsetmacro\valueA{pii(3,3,2,-0.25)}
\pgfmathsetmacro\valueB{pii(3.5,3.5,3,-0.5)}
\pgfmathsetmacro\valueC{pii(1.87868,3,2,-0.25)}

\draw [thick, gray, dashed, fill opacity=0.5]  (axis cs:3,0) -- (axis cs:3,\valueA);
\draw [thick, gray, dashed, fill opacity=0.5]  (axis cs:3.5,0) -- (axis cs:3.5,\valueB);
\draw [thick, gray, dashed, fill opacity=0.5]  (axis cs:1.87868,0) -- (axis cs:1.87868,\valueC);



\node[below] at (axis cs:0.17157, 0)  {$p_N$}; 
\node[below] at (axis cs:1.05051, 0)  {$p_1$}; 
\node[below] at (axis cs:1.87868, 0)  {$\widetilde{p}$}; 
\node[below] at (axis cs:3, 0)  {$P^{\text{olig}}$}; 
\node[below] at (axis cs:3.5, 0)  {$p_M$}; 
\end{axis}
\end{tikzpicture}
\end{center}
\vspace*{-12pt}

\begin{itemize}
    \item For prices in the interval (1), oligopoly profits are positive, but monopoly profits are negative.
    \item For prices in the interval (2), both monopoly and oligopoly profits are positive, but monopoly profits are smaller in comparison. There are no incentives to deviate.
    \item For prices in the interval (3), both monopoly and oligopoly profits are positive, but monopoly profits are higher in comparison. There are  incentives to deviate by doing \textit{undercutting}.
\end{itemize}

\end{solution}
\end{document}